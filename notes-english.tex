\documentclass[12pt, letterpaper, onecolumn, final]{report}
%\documentclass[12pt, letterpaper, onecolumn, draft]{report}

\usepackage[hmargin=1in,vmargin=1in]{geometry}
%\usepackage[inner=1in,outer=1.25in]{geometry}

\usepackage{graphicx} % insert pictures

\usepackage[bookmarks=true,
			colorlinks,
			linkcolor=black,
			anchorcolor=blue,
			citecolor=green
			]{hyperref}
			
\usepackage{indentfirst}
\usepackage{setspace}
%\usepackage[UTF8]{ctex} % Chinese support

%% Cover Page
\title{English Notes}
\author{Jian Tang}
%\date{text}

%% TOC and Abstract Page
\renewcommand{\contentsname}{Table of Content}
\setcounter{tocdepth}{2}
\renewcommand{\abstractname}{Abstract}

%% Normal Page
\linespread{1.67}

\begin{document}
\maketitle
\tableofcontents

\chapter{Tenses}
\section{Present continuous (I am doing)}
1. We use the present continuous when we talk about a change that has started to happen:

Is your English getting better? (not Does your English get better)

The population of the world is increasing very fast. (not increases)

At first I didn’t like my job, but I’m starting to enjoy it now. (not I start)

2. We use the continuous for temporary situations (things that continue for a short time):

I’m living with some friends until I find a place of my own.

a: You’re working hard today.

b: Yes, I have a lot to do.

3. I’m always doing something = I do it too often or more often than normal.

Paul is never satisfied. He’s always complaining. (= he complains too much)

You’re always looking at your phone.  Don’t you have anything else to do? 

\section{Present simple (I do)}
1. We use the simple for things in general or things that happen repeatedly.

\chapter{Usage of Words or Phrases}
\section{as to}
This is a list of positive words. Specifically, 'accurate' is a common such word. \textbf{As to} 'accurate', it is usually used in technical writing.

\section{does not mean that it is easy}
Mozart music from the classical era and composed in rules \textbf{does not mean that it is easy}.

\section{be cast as}

verb: cast

assign a part in a play, film, or other production to (an actor).

verb past participle: cast

He \textbf{was cast as} the Spanish dancer.

\section{be worth}
We hope our attempts to provide you with tools for understanding this field prove to \textbf{be worth} time and effort you spend with our book.

\section{suffer difficulties}
Now I rejoice, and those \textbf{difficulties} that I \textbf{suffered} for so many years make more dear my present blessings.

\chapter{Word Comparing}
\section{essential, essence}
\begin{description}
	\item[essential] a thing that is absolutely necessary.
	\item[essence] the intrinsic nature or indispensable quality of something, especially something abstract, that determines its character.
\end{description}

\end{document}